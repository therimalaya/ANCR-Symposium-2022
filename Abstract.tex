% Options for packages loaded elsewhere
\PassOptionsToPackage{unicode}{hyperref}
\PassOptionsToPackage{hyphens}{url}
\PassOptionsToPackage{dvipsnames,svgnames,x11names}{xcolor}
%
\documentclass[
  11pt,
  a4paper,
]{article}
\usepackage{amsmath,amssymb}
\usepackage[]{libertinus}
\usepackage{setspace}
\usepackage{iftex}
\ifPDFTeX
  \usepackage[T1]{fontenc}
  \usepackage[utf8]{inputenc}
  \usepackage{textcomp} % provide euro and other symbols
\else % if luatex or xetex
  \usepackage{unicode-math}
  \defaultfontfeatures{Scale=MatchLowercase}
  \defaultfontfeatures[\rmfamily]{Ligatures=TeX,Scale=1}
\fi
% Use upquote if available, for straight quotes in verbatim environments
\IfFileExists{upquote.sty}{\usepackage{upquote}}{}
\IfFileExists{microtype.sty}{% use microtype if available
  \usepackage[]{microtype}
  \UseMicrotypeSet[protrusion]{basicmath} % disable protrusion for tt fonts
}{}
\makeatletter
\@ifundefined{KOMAClassName}{% if non-KOMA class
  \IfFileExists{parskip.sty}{%
    \usepackage{parskip}
  }{% else
    \setlength{\parindent}{0pt}
    \setlength{\parskip}{6pt plus 2pt minus 1pt}}
}{% if KOMA class
  \KOMAoptions{parskip=half}}
\makeatother
\usepackage{xcolor}
\IfFileExists{xurl.sty}{\usepackage{xurl}}{} % add URL line breaks if available
\IfFileExists{bookmark.sty}{\usepackage{bookmark}}{\usepackage{hyperref}}
\hypersetup{
  pdftitle={Trends in melanoma tumour thickness in Norway, 1983--2019},
  pdfauthor={Raju Rimal1,; Trude E Robsahm2; Adele Green3,4; Reza Ghiasvand2,5; Corina S Rueegg5; Assia Bassarova6; Petter Gjersvik7; Elisabete Weiderpass8; Odd O Aalen1; Bjørn Møller9; Marit B Veierød1},
  colorlinks=true,
  linkcolor={Maroon},
  filecolor={Maroon},
  citecolor={Blue},
  urlcolor={Blue},
  pdfcreator={LaTeX via pandoc}}
\urlstyle{same} % disable monospaced font for URLs
\usepackage[margin=1in]{geometry}
\usepackage{graphicx}
\makeatletter
\def\maxwidth{\ifdim\Gin@nat@width>\linewidth\linewidth\else\Gin@nat@width\fi}
\def\maxheight{\ifdim\Gin@nat@height>\textheight\textheight\else\Gin@nat@height\fi}
\makeatother
% Scale images if necessary, so that they will not overflow the page
% margins by default, and it is still possible to overwrite the defaults
% using explicit options in \includegraphics[width, height, ...]{}
\setkeys{Gin}{width=\maxwidth,height=\maxheight,keepaspectratio}
% Set default figure placement to htbp
\makeatletter
\def\fps@figure{htbp}
\makeatother
\setlength{\emergencystretch}{3em} % prevent overfull lines
\providecommand{\tightlist}{%
  \setlength{\itemsep}{0pt}\setlength{\parskip}{0pt}}
\setcounter{secnumdepth}{-\maxdimen} % remove section numbering
\ifLuaTeX
  \usepackage{selnolig}  % disable illegal ligatures
\fi

\title{Trends in melanoma tumour thickness in Norway, 1983--2019}
\usepackage{etoolbox}
\makeatletter
\providecommand{\subtitle}[1]{% add subtitle to \maketitle
  \apptocmd{\@title}{\par {\large #1 \par}}{}{}
}
\makeatother
\subtitle{Abstract for ANCR Symposium 2022}
\author{Raju Rimal\textsuperscript{1,*} \and Trude E
Robsahm\textsuperscript{2} \and Adele
Green\textsuperscript{3,4} \and Reza
Ghiasvand\textsuperscript{2,5} \and Corina S
Rueegg\textsuperscript{5} \and Assia
Bassarova\textsuperscript{6} \and Petter
Gjersvik\textsuperscript{7} \and Elisabete
Weiderpass\textsuperscript{8} \and Odd O
Aalen\textsuperscript{1} \and Bjørn Møller\textsuperscript{9} \and Marit
B Veierød\textsuperscript{1}}
\date{Updated on: 30 May, 2022}

\begin{document}
\maketitle

\setstretch{1.25}
\textsuperscript{1} Oslo Centre for Biostatistics and Epidemiology,
Department of Biostatistics, Institute of Basic Medical Sciences,
University of Oslo, Oslo, Norway\\
\textsuperscript{2} Department of Research, Cancer Registry of Norway,
Oslo, Norway\\
\textsuperscript{3} Department of Population Health, QIMR Berghofer
Medical Research Institute, Brisbane, Australia\\
\textsuperscript{4} Cancer Research UK Manchester Institute, University
of Manchester, Manchester, United Kingdom\\
\textsuperscript{5} Oslo Centre for Biostatistics and Epidemiology, Oslo
University Hospital, Oslo, Norway\\
\textsuperscript{6} Department of Pathology, Oslo University
Hospital--Ullevål, Oslo, Norway\\
\textsuperscript{7} Institute of Clinical Medicine, University of Oslo,
Oslo, Norway\\
\textsuperscript{8} International Agency for Research on Cancer, Lyon,
France\\
\textsuperscript{9} Department of Registration, Cancer Registry of
Norway, Oslo, Norway

\textsuperscript{*} Correspondence:
\href{mailto:raju.rimal@medisin.uio.no}{Raju Rimal
\textless{}\href{mailto:raju.rimal@medisin.uio.no}{\nolinkurl{raju.rimal@medisin.uio.no}}\textgreater{}}

\emph{Word count:} 249/250

\hypertarget{abstract}{%
\section{Abstract}\label{abstract}}

Tumour thickness at diagnosis is the most important prognostic factor
for localized primary melanoma. Using thickness data (1980--2007) from
the Cancer Registry of Norway and the Norwegian Melanoma Registry
(2008--2019), we investigated trends in tumour thickness, overall, and
in important subgroups, 1983--2019.

Thickness (mm) was categorized: T1 (\(\le\) 1.0), T2 (1.0-2.0), T3
(\textgreater2.0-4.0), and T4 (\textgreater4.0). Missing was imputed
using multiple-imputation and the incidence rates age-standardized using
the European standard population. Annual percentage change (APC) and
average APC (AAPC) with 95\% confidence intervals (CIs) were estimated.

Age-standardized melanoma incidence increased from 17.7 to 33.3 in women
and 12.9 to 35.2 in men. Men were diagnosed with thicker melanomas than
women. Largest increase was found for T1, AAPC (95\% CI) 3.1 (2.7--3.5)
in women and 4.5 (4.1--4.9) in men, followed by T2 (2.0 (1.6--2.5) and
2.9 (2.5--3.3), respectively) and T4 (0.9 (0.4--1.4) and 1.3 (0.9--1.7),
respectively. A plateau was observed in T1 incidence in women
(1990--2004) and men (1991--2003). In superficial spreading melanoma, a
similar pattern was found for T1 overall. In nodular melanomas, T3 is
dominating and has a fluctuating trend in women and an increasing trend
that seems to stabilize in men. In T4, an increasing trend is seen in
women and a fluctuating trend in men.

T1 melanomas had the largest increase in incidence. An increasing trend
was also observed in thicker tumours, suggesting that the rise in
melanoma incidence is not only due to overdiagnosis/pathological
practice.

\end{document}
